\documentclass[9pt,twocolumn,twoside,]{pnas-new}

%% Some pieces required from the pandoc template
\providecommand{\tightlist}{%
  \setlength{\itemsep}{0pt}\setlength{\parskip}{0pt}}

% Use the lineno option to display guide line numbers if required.
% Note that the use of elements such as single-column equations
% may affect the guide line number alignment.


\usepackage[T1]{fontenc}
\usepackage[utf8]{inputenc}



\templatetype{pnasresearcharticle}  % Choose template

\title{Template for preparing your research report submission to PNAS using
RMarkdown}

\author[a,1,2]{Alice Anonymous}
\author[a,b]{Bob Security}

  \affil[a]{Some Institute of Technology, Department, Street, City, State, Zip}
  \affil[b]{Another University Department, Street, City, State, Zip}


% Please give the surname of the lead author for the running footer
\leadauthor{Anonymous}

% Please add here a significance statement to explain the relevance of your work
\significancestatement{Authors must submit a 120-word maximum statement about the significance
of their research paper written at a level understandable to an
undergraduate educated scientist outside their field of speciality. The
primary goal of the Significance Statement is to explain the relevance
of the work in broad context to a broad readership. The Significance
Statement appears in the paper itself and is required for all research
papers.}


\authorcontributions{Please provide details of author contributions here.}

\authordeclaration{Please declare any conflict of interest here.}

\equalauthors{\textsuperscript{} }

\correspondingauthor{\textsuperscript{} }

% Keywords are not mandatory, but authors are strongly encouraged to provide them. If provided, please include two to five keywords, separated by the pipe symbol, e.g:
 \keywords{  one |  two |  optional |  optional |  optional  } 

\begin{abstract}
Please provide an abstract of no more than 250 words in a single
paragraph. Abstracts should explain to the general reader the major
contributions of the article. References in the abstract must be cited
in full within the abstract itself and cited in the text.
\end{abstract}

\dates{This manuscript was compiled on \today}
\doi{\url{www.pnas.org/cgi/doi/10.1073/pnas.XXXXXXXXXX}}

\begin{document}

% Optional adjustment to line up main text (after abstract) of first page with line numbers, when using both lineno and twocolumn options.
% You should only change this length when you've finalised the article contents.
\verticaladjustment{-2pt}

\maketitle
\thispagestyle{firststyle}
\ifthenelse{\boolean{shortarticle}}{\ifthenelse{\boolean{singlecolumn}}{\abscontentformatted}{\abscontent}}{}

% If your first paragraph (i.e. with the \dropcap) contains a list environment (quote, quotation, theorem, definition, enumerate, itemize...), the line after the list may have some extra indentation. If this is the case, add \parshape=0 to the end of the list environment.

\acknow{Please include your acknowledgments here, set in a single paragraph.
Please do not include any acknowledgments in the Supporting Information,
or anywhere else in the manuscript.}

\correspondingauthor{\textsuperscript{1}To whom correspondence should be addressed. E-mail: danovan@uw.edu}

This PNAS journal template is provided to help you write your work in
the correct journal format. Instructions for use are provided below.

Note: please start your introduction without including the word
``Introduction'' as a section heading (except for math articles in the
Physical Sciences section); this heading is implied in the first
paragraphs.

\hypertarget{guide-to-using-this-template}{%
\section*{Guide to using this
template}\label{guide-to-using-this-template}}
\addcontentsline{toc}{section}{Guide to using this template}

Please note that whilst this template provides a preview of the typeset
manuscript for submission, to help in this preparation, it will not
necessarily be the final publication layout. For more detailed
information please see the
\href{http://www.pnas.org/site/authors/format.xhtml}{PNAS Information
for Authors}.

\hypertarget{author-affiliations}{%
\subsection*{Author Affiliations}\label{author-affiliations}}
\addcontentsline{toc}{subsection}{Author Affiliations}

Include department, institution, and complete address, with the
ZIP/postal code, for each author. Use lower case letters to match
authors with institutions, as shown in the example. Authors with an
ORCID ID may supply this information at submission.

\hypertarget{submitting-manuscripts}{%
\subsection*{Submitting Manuscripts}\label{submitting-manuscripts}}
\addcontentsline{toc}{subsection}{Submitting Manuscripts}

All authors must submit their articles at
\href{http://www.pnascentral.org/cgi-bin/main.plex}{PNAScentral}. If you
are using Overleaf to write your article, you can use the ``Submit to
PNAS'' option in the top bar of the editor window.

\hypertarget{format}{%
\subsection*{Format}\label{format}}
\addcontentsline{toc}{subsection}{Format}

Many authors find it useful to organize their manuscripts with the
following order of sections; Title, Author Affiliation, Keywords,
Abstract, Significance Statement, Results, Discussion, Materials and
methods, Acknowledgments, and References. Other orders and headings are
permitted.

\hypertarget{manuscript-length}{%
\subsection*{Manuscript Length}\label{manuscript-length}}
\addcontentsline{toc}{subsection}{Manuscript Length}

PNAS generally uses a two-column format averaging 67 characters,
including spaces, per line. The maximum length of a Direct Submission
research article is six pages and a PNAS PLUS research article is ten
pages including all text, spaces, and the number of characters displaced
by figures, tables, and equations. When submitting tables, figures,
and/or equations in addition to text, keep the text for your manuscript
under 39,000 characters (including spaces) for Direct Submissions and
72,000 characters (including spaces) for

\showmatmethods
\showacknow
\pnasbreak



% Bibliography
% \bibliography{pnas-sample}

\end{document}

